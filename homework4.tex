\documentclass[12pt]{exam}

\usepackage{amssymb,amsfonts,amsmath}
\usepackage[letterpaper,margin=1in]{geometry}
\usepackage{graphicx}
\usepackage[numbered]{matlab-prettifier}

\newcommand{\class}{MATH 375}
\newcommand{\term}{Fall 2016}
\newcommand{\doctitle}{Homework 4}

\parindent 0ex

\pagestyle{head}
\header{\bf \class}{\bf \doctitle\ - Page \thepage\ of \numpages}{\bf \term}
\headrule

%\renewcommand{\arraystretch}{1.5}

\begin{document}

Remember to adequately label all plots and include any requested code listings with your solutions. \emph{Only include those scripts and functions which are requested}. A clear and complete presentation of your solutions is required for full credit.

\begin{questions}

\question Use the provided function \verb$backsub$ to solve the system $A \mathbf{x} = \mathbf{b}$ where $A \in \mathbb{R}^{10\times 10}$ with
\[a_{ij} = \begin{cases} \cos(ij) & i \leq j \\ 0 & i > j \end{cases}\]
and $\mathbf{b} \in \mathbb{R}^{10}$ with $b_i = \tan(i)$.

\question A system $A \mathbf{x} = \mathbf{b}$ is called lower triangular if $a_{ij} = 0$ when $i < j$.

\begin{parts}
\part Write a function \verb$forsub$, analogous to \verb$backsub$, to solve a lower triangular system. Include a listing of your function.
\part Use your function \verb$forsub$ to solve the system $A \mathbf{x} = \mathbf{b}$ where $A \in \mathbb{R}^{20\times 20}$ with
\[a_{ij} = \begin{cases} i+j & i \geq j \\ 0 & i < j \end{cases}\]
and $\mathbf{b} \in \mathbb{R}^{20}$ with $b_i = i$.
\end{parts}

\question In Gaussian elimination, selecting a row below a pivot and swapping rows is called pivoting. A pivoting strategy gives a rule for determining when to swap rows and which row to swap with. We will consider two pivoting strategies:

\begin{description}
\item[``Trivial'' Pivoting.] If $a_{pp} \neq 0$, do not swap rows. If $a_{pp} = 0$, locate the first row below $p$ in which $a_{kp} \neq 0$ and interchange rows $k$ and $p$. This will result in a new element $a_{pp} \neq 0$, which is a nonzero pivot.
\item[Partial Pivoting.] Find $k \geq p$ so that $|a_{kp}|$ is maximum. If there are multiple entries with the same maximum value, select the one with smallest $k$. Interchange rows $k$ and $p$ if necessary.
\end{description}

For each strategy, complete the following two tasks.

\begin{parts}
\part Modify the provided function \verb$gausselim$ to implement the pivoting strategy. Include a listing of your function.
\part Use your modified function to determine an equivalent upper triangular system to $A \mathbf{x} = \mathbf{b}$ with
\[A = \begin{pmatrix} 1 & 2 & 3 & 4 \\ 1 & 2 & 4 & 6 \\ 1 & 3 & 6 & 6 \\ 1 & 4 & 4 & 7 \end{pmatrix} \quad\text{and}\quad b = \begin{pmatrix} 1 \\ 1 \\ 1 \\ 1 \end{pmatrix}\]
\end{parts}

\newpage

\question The Hilbert matrix is a classical ill-conditioned matrix, and so small changes in its coefficients will produce a large change in the solution to the perturbed system.

\begin{parts}
\part Find the exact solution of $A\mathbf{x} = \mathbf{b}$, leaving all numbers as fractions and performing exact arithmetic, using the $4 \times 4$ Hilbert matrix
\[A = \begin{pmatrix} 1 & 1/2 & 1/3 & 1/4 \\ 1/2 & 1/3 & 1/4 & 1/5 \\ 1/3 & 1/4 & 1/5 & 1/6 \\ 1/4 & 1/5 & 1/6 & 1/7 \end{pmatrix} \quad\text{and}\quad b = \begin{pmatrix} 1 \\ 0 \\ 0 \\ 0 \end{pmatrix}\]
\part Find the approximate solution of $A\mathbf{x} = \mathbf{b}$, rounding to four digits after each step, using the $4 \times 4$ approximate Hilbert matrix
\[A = \begin{pmatrix} 1.0000 & 0.5000 & 0.3333 & 0.2500 \\ 0.5000 & 0.3333 & 0.2500 & 0.2000 \\ 0.3333 & 0.2500 & 0.2000 & 0.1667 \\ 0.2500 & 0.2000 & 0.1667 & 0.1429 \end{pmatrix} \quad\text{and}\quad b = \begin{pmatrix} 1 \\ 0 \\ 0 \\ 0 \end{pmatrix}\]
Comment on the difference between your two results.
\end{parts}

\question Consider the problem of trying to find the degree 6 polynomial $y = c_0 + c_1 x + c_2 x^2 + c_3 x^3 + c_4 x^4 + c_5 x^5 + c_6 x^6$ passing through the points $(0,1)$, $(1,3)$, $(2,2)$ $(3,1)$, $(4,3)$, $(5,2)$, and $(6,1)$.
\begin{parts}
\part Write down a linear system that you can solve for the coefficients $c_0$, $c_1$, $c_2$, $c_3$, $c_4$, $c_5$, and $c_6$.
\part Solve the linear system from part (a) using the provided \verb$gausselim$ and \verb$backsub$ functions. Plot the polynomial and the given points on the same graph and comment on any discrepancies.
\part Repeat part (b) using the function you wrote earlier implementing Gaussian elimination with partial pivoting instead of \verb$gausselim$.
\end{parts}

\end{questions}

\end{document}
