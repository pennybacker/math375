\documentclass[12pt,fleqn]{exam}

\usepackage{amssymb,amsfonts,amsmath}
\usepackage[letterpaper,margin=1in]{geometry}
\usepackage{graphicx}
\usepackage{tabularx}

\newcommand{\class}{MATH/CS 375}
\newcommand{\term}{Spring 2017}
\newcommand{\doctitle}{Homework 1}

\newcommand{\R}{\ensuremath{\mathbb{R}}}
\newcommand{\fl}{\ensuremath{\operatorname{fl}}}

\parindent 0ex

\pagestyle{head}
\header{\bf \class}{\bf \doctitle\ - Page \thepage\ of \numpages}{\bf \term}
\headrule

\begin{document}

\begin{questions}

\question (Sauer \S0.2, \#4) Convert the following base 10 numbers to binary. Use overbar notation for nonterminating binary numbers.
\begin{parts}
\part 11.25
\part 2/3
\part 3/5
\part 3.2
\part 30.6
\part 99.9
\end{parts}

\question (Sauer \S0.2, \#8) Convert the following binary numbers to base 10.
\begin{parts}
\part 11011
\part 110111.001
\part $111.\overline{001}$
\part $1010.\overline{01}$
\part $10111.1\overline{0101}$
\part $1111.010\overline{001}$
\end{parts}

\question (Sauer \S0.3, \#2) Convert the following base 10 numbers to binary and express each as a floating point number $\fl(x)$ by using the Rounding to Nearest Rule.
\begin{parts}
\part 9.5
\part 9.6
\part 100.2
\part 44/7
\end{parts}

\question (Sauer \S0.3, \#4) Find the largest integer $k$ for which $\fl(19 + 2^{-k}) > \fl(19)$ in double precision floating point arithmetic.

\question (Sauer \S0.3, \#11) Does the associative law hold for IEEE computer addition? Explain your response.

\question (Sauer \S0.4, \#1) Identify for which values of $x$ there is subtraction of nearly equal numbers, and find an alternate form that avoids the problem.

\begin{parts}
\part $\displaystyle\frac{1-\sec^2(x)}{\tan(x)}$
\part $\displaystyle\frac{1-(1-x)^3}{x}$
\part $\displaystyle\frac{1}{1+x}-\frac{1}{1-x}$
\end{parts}

\end{questions}

\end{document}
