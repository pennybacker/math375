\documentclass[12pt]{exam}

\usepackage{amssymb,amsfonts,amsmath}
\usepackage[letterpaper,margin=1in]{geometry}
\usepackage{graphicx}
\usepackage{fancyvrb}

\newcommand{\class}{MATH 375}
\newcommand{\term}{Fall 2016}
\newcommand{\doctitle}{Homework 1}

\parindent 0ex

\pagestyle{head}
\header{\bf \class}{\bf \doctitle\ - Page \thepage\ of \numpages}{\bf \term}
\headrule

\renewcommand{\arraystretch}{1.5}

\begin{document}

Remember to adequately label all plots and include any requested Matlab and functions with your solutions. \emph{Only include those scripts and functions which are requested}. A clear and complete presentation of your solutions is required for full credit.
\begin{questions}

\question Suppose \verb$z = [10,40,50,80,30,70,60,90]$. What does this vector look like after each of the following commands? Explain in words what each command does. Do not reinitialize \verb$z$ between commands.

\begin{parts}
\part \verb$z(1:2:7) = zeros(1,4)$

\part \verb$z(7:-2:1) = fliplr(z(1:2:7))$

\part \verb$z([3,4,8,1]) = []$

\part \verb$z(z > 50) = 10$

\part \verb$z(find(z)) = 0$
\end{parts}

\question Use the \verb$linspace$ function to create vectors identical to each of the following vectors created with colon notation.

\begin{parts}
\part \verb$t = 5:5:30$

\part \verb$x = -3:3$
\end{parts}

\question Use colon notation to create vectors identical to each of the following vectors created with the \verb$linspace$ function.
\begin{parts}
\part \verb$v = linspace(-2,1,5)$

\part \verb$r = linspace(6,0,7)$
\end{parts}

\question Let \verb$a$ be an $n \times n$ matrix. For each of the following Matlab commands, determine the asymptotic runtime as $n$ gets large by measuring the time it take to execute each command at a variety of values of $n$ (say $n = 100, 110, \dots, 1000$). You will need to average over many executions in order to get meaningful data. Include a log-log plot of runtime versus $n$ which demonstrates your solution for each command.

\begin{parts}
\part \verb$trace(a)$

\part \verb$det(a)$

\part \verb$inv(a)$
\end{parts}

\question As we will see later in the semester, the derivative $f'(x)$ of a differentiable function $f(x)$ can be approximated by a \emph{finite difference} formula. Two such formulas are the forward difference
\[f'(x) \approx \frac{f(x+h) - f(x)}{h}\]
and the centered difference
\[f'(x) \approx \frac{f(x+h) - f(x-h)}{2h}\]
where $h$ is a (small) positive real number.

\begin{parts}
\part Write two Matlab functions with definitions \verb$fp = fdiff(f,x,h)$ and \\ \verb$fp = cdiff(f,x,h)$ implementing the forward difference formula and the centered difference formula respectively. Include a listing of each function in your solutions. In both cases, the input \verb$f$ is an arbitrary function passed using the \verb$@$ notation, \verb$x$ is a vector of values at which to approximate the derivative, and \verb$h$ is a positive scalar. The output \verb$fp$ is an approximation to the derivative. Do not use a \verb$for$ loop in your functions.

\part Use your functions from part (a) to approximate the derivative of $f(x) = 1/(1+x^2)$ on the interval $[-1,1]$. Take \verb$x = linspace(-1,1,100)$ and \verb$h = 1e-4$. Prepare the following plot having two subplots by using the built-in commands \verb$subplot(2,1,1)$ and \verb$subplot(2,1,2)$. The top plot should depict the absolute value of the error between the forward difference approximation and the exact derivative. The bottom plot should depict the absolute value of the error between the centered difference approximation and the exact derivative.

\part Now, take $f(x) = e^x$ and $h = 10^{-1}, 10^{-2}, \dots, 10^{-9}$. Approximate $f'(0)$ using the two finite difference formulas from part (a) at each value of $h$. The exact value is obviously $f'(0) = 1$. Plot the absolute errors from each method versus $h$ on the same log-log plot. Your plot should include a legend. What order of convergence do you observe for each method? Can you explain what is happening for the smallest values of $h$?
\end{parts}

\end{questions}

\end{document}
