\documentclass[12pt]{exam}

\usepackage{amssymb,amsfonts,amsmath}
\usepackage[letterpaper,margin=1in]{geometry}
\usepackage{graphicx}

\newcommand{\class}{MATH 375}
\newcommand{\term}{Spring 2015}
\newcommand{\doctitle}{Homework 3}

\newcommand{\matlab}{{\sc Matlab}}

\parindent 0ex

\pagestyle{head}
\header{\bf \class}{\bf \doctitle\ - Page \thepage\ of \numpages}{\bf \term}
\headrule

\renewcommand{\arraystretch}{1.5}

\begin{document}

Remember to adequately label all plots and include any \matlab~scripts and functions with your solutions. A clear and complete presentation of your solutions is required for full credit.
\begin{questions}

\question The roots of the quadratic equation $ax^2+bx+c = 0$ are
\[x_\pm = \frac{-b \pm \sqrt{b^2-4ac}}{2a} .\]
We may also approximate these roots using fixed-point iteration by dividing through by $x$ (assuming $x \neq 0$) to get the equivalent equation
\[ax+b+\frac{c}{x} = 0.\]
Now, we can rearrange this equation to get the following methods:
\begin{parts}
\part Method 1 (Forward Iteration): Consider the sequence
\[x_0 = \text{initial guess}, \quad x_{k+1} = -\frac{b}{a}-\frac{c}{ax_k}, k = 1,2,3,\dots\]
\part Method 2 (Backward Iteration): Consider the sequence
\[x_0 = \text{initial guess}, \quad x_{k+1} = -\frac{c}{b+ax_k}, k = 1,2,3,\dots\]
\end{parts}
Write \matlab~functions \verb$forward$ and \verb$backward$ that implement the above iterations to return $r$ which differs from either $x_+$ or $x_-$ by no more than a tolerance \verb$tol$. Your functions should have \verb$(a,b,c,x0,tol)$ as their argument lists and terminate when $|x_{k+1}-x_k| < \text{\texttt{tol}}$. Demonstrate the performance of your functions for the values
\[x0 = 1,\ a = 1,\ b = c = -1,\ \text{\texttt{tol}} = 10^{-10}.\]
Note that the two functions will return different approximate roots $r_\text{forward}$ and $r_\text{backward}$. Verify directly that each is within the desired tolerance from $x_\pm$. Which root is found by each method? Does the answer change if you change $x_0$?

\question Use the bisection method and the \matlab~function \verb$fzero$ to compute a positive real number $x$ satisfying $\sinh x = \cos x$. For each method, use a tolerance of $10^{-8}$. List your initial approximation (or interval in the case of bisection) and the number of required iterations. Also print at least nine digits of the approximate roots.

\question Use the bisection method and the \matlab~function \verb$fzero$ to compute all three real numbers $x$ satisfying $5x^2-e^x = 0$. For each of the three roots and each method, use a tolerance of $10^{-8}$. List your initial approximations (or intervals in the case of bisection) and the number of required iterations. Also print at least nine digits of each approximate roots.

\question Sauer, Section 1.2, Exercise 14.

\end{questions}

\end{document}
