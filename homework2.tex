\documentclass[12pt,fleqn]{exam}

\usepackage{amssymb,amsfonts,amsmath}
\usepackage[letterpaper,margin=1in]{geometry}
\usepackage{graphicx}
\usepackage{tabularx}

\newcommand{\class}{MATH/CS 375}
\newcommand{\term}{Spring 2017}
\newcommand{\doctitle}{Homework 2}

\newcommand{\R}{\ensuremath{\mathbb{R}}}
\newcommand{\fl}{\ensuremath{\operatorname{fl}}}

\parindent 0ex

\pagestyle{head}
\header{\bf \class}{\bf \doctitle\ - Page \thepage\ of \numpages}{\bf \term}
\headrule

\begin{document}


Computer problems contain ``CP'' in the problem number. For these problems, remember to adequately label all plots and include any code that you have written along with your solutions. Do all other problems by hand and make sure to your work. A clear and complete presentation of your solutions is required for full credit.

\begin{questions}

\question (Sauer \S0.4, CP1) Use Matlab to calculate the expressions that follow in double precision arithmetic for $x = 10^{-1}, \dots, 10^{-14}$. Then, using an alternative form of the expression that doesn't suffer from subtracting nearly equal numbers, repeat the calculation and make a table of results. Report the number of correct digits in the original expression for each $x$.

\begin{parts}
\part $\displaystyle\frac{1 - \sec(x)}{\tan^2(x)}$
\part $\displaystyle\frac{1 - (1 - x)^3}{x}$
\end{parts}

\question (Sauer \S0.4, CP2) Find the limit of the expression as $x \rightarrow 0$, and then use Matlab to find the smallest value of $p$ for which the expression calculated in double precision arithmetic at $x = 10^{-p}$ has no correct significant digits.

\begin{parts}
\part $\displaystyle\frac{\tan(x) - x}{x^3}$
\part $\displaystyle\frac{e^x + \cos(x) - \sin(x) - 2}{x^3}$
\end{parts}

\question (Sauer \S1.1, \#2, \#4) Use the Intermediate Value Theorem to find an interval of length one that contains a root of the equation. Apply two steps of the Bisection Method to find an approximate root within 1/8 of the true root.

\begin{parts}
\part $\displaystyle x^5 + x = 1$
\part $\displaystyle \sin(x) = 6x + 5$
\part $\displaystyle \ln(x) + x^2 = 3$
\end{parts}

\question (Sauer \S1.1, \#5) Consider the equation $x^4 = x^3 + 10$.

\begin{parts}
\part Find an interval $[a,b]$ of length one inside which the equation has a solution.
\part Starting with $[a,b]$, how many steps of the Bisection Method are required to calculate the
solution within $10^{-10}$? Answer with an integer.
\end{parts}

\question (Sauer \S1.1, CP3) Use the Bisection Method to locate all solutions of the following equations. Sketch the function by using Matlab's \verb$plot$ command and identify three intervals of length one that contain a root. Then find the roots to six correct decimal places.

\begin{parts}
\part $\displaystyle 2x^3 - 6x - 1 = 0$
\part $\displaystyle e^{x-2} + x^3 - x = 0$
\part $\displaystyle 1 + 5x - 6x^3 - e^{2x} = 0$
\end{parts}

\question (Sauer \S1.1, CP7) Use the Bisection Method to find the two real numbers $x$, within six correct decimal places, that make the determinant of the matrix
\[A = \begin{bmatrix}
1 & 2 & 3 & x \\
4 & 5 & x & 6 \\
7 & x & 8 & 9 \\
x & 10 & 11 & 12
\end{bmatrix}\]
equal to 1000. For each solution you find, test it by computing the corresponding determinant and reporting how many correct decimal places (after the decimal point) the determinant has when your solution $x$ is used. You may use the Matlab command \verb$det$ to compute the determinants.

\end{questions}

\end{document}
