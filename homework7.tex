\documentclass[12pt,answers]{exam}

\usepackage{amssymb,amsfonts,amsmath}
\usepackage[letterpaper,margin=1in]{geometry}
\usepackage{graphicx}
\usepackage[numbered]{matlab-prettifier}

\renewcommand*{\vec}[1]{\boldsymbol{#1}}

\newcommand{\class}{MATH 375}
\newcommand{\term}{Fall 2016}
\newcommand{\doctitle}{Homework 7}

\parindent 0ex

\pagestyle{head}
\header{\bf \class}{\bf \doctitle\ - Page \thepage\ of \numpages}{\bf \term}
\headrule

%\renewcommand{\arraystretch}{1.5}

\begin{document}

Remember to adequately label all plots and include any requested code listings with your solutions. \emph{Only include those scripts and functions which are requested}. Show your work for problems that you do by hand. A clear and complete presentation of your solutions is required for full credit.

\begin{questions}

\question \emph{Polynomial Interpolation}. By hand, find the unique polynomial $P(x)$ having degree at most two that passes through the three points $(-1,1)$, $(2,3)$, $(3,0)$ using each of the following three approaches.

\begin{parts}
\part Vandermonde matrix.
\part Lagrange's interpolating polynomials.
\part Newton's divided differences.
\end{parts}

Verify that the polynomial you obtain is the same in (a)--(c).

\question \emph{Change of Basis}. As we have seen, each approach to polynomial interpolation yields an expression for the interpolating polynomial $P(x)$ as a linear combination of basis functions: specifically $\{1,x,x^2,\dots,x^{n-1}\}$ for the Vandermonde matrix approach and $\{\ell_1(x),\ell_2(x),\dots,\ell_n(x)\}$ for the Lagrange interpolating polynomial approach.

Find a linear transformation which can be used to change between these bases. That is, find a matrix $M$ so that
\[M \begin{bmatrix} \ell_1(x) \\ \ell_2(x) \\ \vdots \\ \ell_n(x) \end{bmatrix} = \begin{bmatrix} 1 \\ x \\ \vdots \\ x^{n-1} \end{bmatrix}.\]

\question \emph{Newton's Divided Differences}. You can use the included script \verb$homework7_test.m$ to test the functions that you write for this problem.
\begin{parts}
\part Recall that the unique polynomial of degree at most $n-1$ interpolating $n$ points $x_1 < x_2 < \dots < x_n$ can be written as
\[P(x) = \sum_{i = 1}^n c_i\,p_i(x) \quad\text{where}\quad p_i(x) = \prod_{k = 1}^{i-1} (x-x_k)\]
and the coefficients are given by the divided differences $c_i = f[x_1,x_2,\dots,x_i]$.
Write a function with the declaration \verb$function c = newton_coef(x,y)$ which computes the coefficients of the Newton polynomials. Its inputs \verb$x$ and \verb$y$ are vectors containing the $x$- and $y$-coordinates of the data to be interpolated and its output \verb$c$ is a vector containing the coefficients $(c_1,c_2,\dots,c_n)$. Include a listing of your function.
\part Write a function with the declaration \verb$function yy = newton_eval(c,x,xx)$ which uses the nested formula
\[P(x) = c_1 + (x-x_1)\big[c_2 + (x-x_2)\big[c_3 + (x-x_3)\big[c_4 + \dots + (x-x_{n-1})\,c_n\big]\big]\big]\]
to evaluate the interpolating polynomial. Its inputs are the vector \verb$c$ of coefficients, a vector \verb$x$ containing the $x$-coordinates of the data points, and a vector \verb$xx$ of points at which to evaluate the polynomial. Its output \verb$yy$ is a vector containing the values of the polynomial. Include a listing of your function.
\part Use the functions you have written to evaluate the polynomial that interpolates $f(x) = \sin(x)$ at 5 equally spaced points on the interval $-\pi \leq x \leq \pi$. Plot the function and its interpolating polynomial on this interval. Both of these should be smooth functions. Also, plot the interpolation error on this interval.
\part Analytically determine an upper bound for the interpolation error at $x = \pi/4$. Does this match your calculations?
\end{parts}

\end{questions}

\end{document}
