\documentclass[12pt]{exam}

\usepackage{amssymb,amsfonts,amsmath}
\usepackage[letterpaper,margin=1in]{geometry}
\usepackage{graphicx}
\usepackage{tikz}
\usepackage{verbatim}

\newcommand{\class}{MATH 375}
\newcommand{\term}{Spring 2015}
\newcommand{\doctitle}{Homework 9}

\newcommand{\matlab}{{\sc Matlab}}

\parindent 0ex

\pagestyle{head}
\header{\bf \class}{\bf \doctitle\ - Page \thepage\ of \numpages}{\bf \term}
\headrule

\newcommand*\circled[1]{\textcircled{\scriptsize{#1}}}

\begin{document}

\vspace{1ex} In addition to calculations and proofs, please hand in neatly formatted output (figures and numbers) from your programs. Hand in code unless otherwise noted, but it should be clear from the output if the program works or not.

\begin{questions}

\question \emph{Numerical Differentiation}.
A second order approximation to $f''(x)$ at $x = x_0$ is given by
\[f''(x_0) \approx \frac{f(x_0+h) - 2f(x_0) + f(x_0-h)}{h^2}.\]
\begin{parts}
\part Suppose that $p(x)$ is the polynomial interpolating $f(x)$ at $x = x_0-h$, $x_0$, and $x_0+h$. Show that the approximation above can be found by evaluating $p''(x_0)$.
\part Substitute the expressions given by Taylor's remainder theorem for $f(x_0-h)$ and $f(x_0+h)$ to prove that this approximation is second order. Find an upper bound for the absolute error term $|f''(x_0)-p''(x_0)|$ given that $|f^{(4)}(x)| < M$ for $x_0-h < x < x_0+h$.
\part Write a program which verifies empirically that the error of this approximation is $\mathcal{O}(h^2)$. You should include a description of your program and a plot of the error which verifies the order.
\end{parts}

\question \emph{Newton-Cotes Quadrature}.
\begin{parts}
\part Write two functions with the following declarations:
\begin{verbatim}
    function i = trapezoid(f,a,b,m)
    function i = midpoint(f,a,b,m)
\end{verbatim}
These implement the trapezoid rule and midpoint rule respectively for approximating the integral
\[I = \int_a^b f(x)\,dx\]
on $m$ subintervals. The input \verb$f$ is a function handle for the integrand which can accept a vector as input, \verb$a$ and \verb$b$ are the bounds of the integral, and \verb$m$ is the number of subintervals to use for the approximation. The output \verb$i$ is an approximation to the integral.

\emph{Important}: Do not use a \verb$for$ loop in any of your functions. You may, however, use vector operations.
\part Approximate the integral $I$ for $[a,b] = [-1,1]$ and 
$f(x) = \frac{1}{1+x^2}$
using each of these methods, taking $h = 1/2, 1/4, 1/8, 1/16, 1/32, 1/64$. Show the error for the methods as a function of $h$ on a single \verb$loglog$ plot. You can find the true solution analytically or by approximating the integral with a very small value of $h$. Hand in your plot and comment on the order of accuracy and cost of the methods.
\part Repeat part (b) for $f(x) = \exp(\sin(6\pi x))$. Comment on your results.
\end{parts}

\end{questions}

\end{document}
