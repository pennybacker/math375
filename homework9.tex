\documentclass[12pt,answers]{exam}

\usepackage{amssymb,amsfonts,amsmath}
\usepackage[letterpaper,margin=1in]{geometry}
\usepackage{graphicx}
\usepackage[numbered]{matlab-prettifier}

\renewcommand*{\vec}[1]{\boldsymbol{#1}}

\newcommand{\class}{MATH 375}
\newcommand{\term}{Fall 2016}
\newcommand{\doctitle}{Homework 9}

\parindent 0ex

\pagestyle{head}
\header{\bf \class}{\bf \doctitle\ - Page \thepage\ of \numpages}{\bf \term}
\headrule

%\renewcommand{\arraystretch}{1.5}

\begin{document}

Remember to adequately label all plots and include any requested code listings with your solutions. \emph{Only include those scripts and functions which are requested}. Show your work for problems that you do by hand. A clear and complete presentation of your solutions is required for full credit.

\begin{questions}

\question \emph{QR Factorization and Least Squares}.
Given an $n \times m$ matrix $A$ with $n \geq m$, you can decompose $A$ into an $n \times m$ matrix $Q_1$ with orthonormal columns and an $m \times m$ upper triangular matrix $R$ as $A = Q_1 R$. We called this the thin $QR$ factorization.
\begin{parts}
\part By hand, compute the factors $Q_1$ and $R$ for the following matrix:
\[A = \begin{bmatrix} 1 & 1 \\ 0 & 1 \\ 1 & 1 \end{bmatrix}\]
You can use your result to test the function \verb$gram_schmidt$ that you implement below.
\part Write a function with the declaration \verb$function [Q,R] = gram_schmidt(A)$ which computes the thin $QR$ factorization of the given matrix using Gram-Schmidt orthogonalization. Its input \verb$A$ is an $n \times m$ matrix $A$ with $n \geq m$. Its outputs \verb$Q$ and \verb$R$ are the $n \times m$ matrix $Q_1$ and the $m \times m$ matrix $R$ with $A = Q_1R$ respectively.
\part The daily high temperature in Albuquerque is roughly modeled by a function of the form
\[y(x) = a_1 + a_2 \cos(2\pi x/12) + a_3 \sin(2\pi x/12) ,\]
where $x$ is measured in months since January and $y$ is measured in degrees Fahrenheit. Data from the National Weather Service, averaged over the years 1914 to 2005, give the following average high temperatures:

\vspace{1ex}
\begin{centering}
\begin{tabular}{|l||l|l|l|l|l|l|l|l|l|l|l|l|}\hline
$x$ & 0 & 1 & 2 & 3 & 4 & 5 & 6 & 7 & 8 & 9 & 10 & 11 \\\hline
$y$ & 47.2 & 53.2 & 60.6 & 70.0 & 79.4 & 89.3 & 91.7 & 88.9 & 82.4 & 71.0 & 56.9 & 47.7 \\\hline
\end{tabular}
\end{centering}
\vspace{1ex}

What is the overdetermined system of equations that relates the model to the data?
\part Use the thin $QR$ factorization given by your function \verb$gram_schmidt$ to solve the least squares problem associated with this overdetermined system for the coefficients $a_1$, $a_2$, and $a_3$. Plot the data and the fitted model on the same axes. What do you predict will be the high temperature on April 15 ($x = 3.5$) using this fit?
\end{parts}

\question \emph{Numerical Differentiation}.
We have so far only discussed numerical approximations of the first derivative, but it is just as reasonable to approximate higher derivatives. For instance, a second order approximation to $f''(x)$ at $x = x_0$ is given by
\[f''(x_0) \approx \frac{f(x_0+h) - 2f(x_0) + f(x_0-h)}{h^2}.\]
\begin{parts}
\part Suppose that $p(x)$ is the polynomial interpolating $f(x)$ at $x = x_0-h$, $x_0$, and $x_0+h$. Show that the approximation above can be found by evaluating $p''(x_0)$.
\part Substitute the expressions given by Taylor's remainder theorem for $f(x_0-h)$ and $f(x_0+h)$ to prove that this approximation is second order. Find an upper bound for the absolute error term $|f''(x_0)-p''(x_0)|$ given that $|f^{(4)}(x)| < M$ for $x_0-h < x < x_0+h$.
\end{parts}

\question \emph{Newton-Cotes Quadrature}.
\begin{parts}
\part Write three functions with the following declarations:
\begin{verbatim}
    function i = trapezoid(f,a,b,m)
    function i = midpoint(f,a,b,m)
    function i = simpsons(f,a,b,m)
\end{verbatim}
These implement the trapezoid rule, midpoint rule, and Simpson's rule respectively for approximating the integral
\[I = \int_a^b f(x)\,dx\]
on $m$ subintervals. The input \verb$f$ is a function handle for the integrand which can accept a vector as input, \verb$a$ and \verb$b$ are the bounds of the integral, and \verb$m$ is the number of subintervals to use for the approximation. The output \verb$i$ is an approximation to the integral.

\emph{Important}: Do not use a \verb$for$ loop in any of your functions. You may, however, use vector operations.
\part Approximate the integral $I$ for $[a,b] = [-1,1]$ and 
$f(x) = \frac{1}{1+x^2}$
using each of these three methods, taking $h = 1/2, 1/4, 1/8, 1/16, 1/32, 1/64$. Show the error for the methods as a function of $h$ on a single \verb$loglog$ plot. You can find the true solution analytically. Hand in your plot and comment on the order of accuracy and cost of the methods.
\part Repeat part (b) for $f(x) = \exp(\sin(6\pi x))$. Comment on your results.
\end{parts}


\end{questions}

\end{document}
