\documentclass[12pt]{exam}

\usepackage{amssymb,amsfonts,amsmath}
\usepackage[letterpaper,margin=1in]{geometry}
\usepackage{graphicx}
\usepackage[numbered]{matlab-prettifier}

\renewcommand*{\vec}[1]{\boldsymbol{#1}}

\newcommand{\class}{MATH 375}
\newcommand{\term}{Fall 2016}
\newcommand{\doctitle}{Homework 6}

\parindent 0ex

\pagestyle{head}
\header{\bf \class}{\bf \doctitle\ - Page \thepage\ of \numpages}{\bf \term}
\headrule

%\renewcommand{\arraystretch}{1.5}

\begin{document}

Remember to adequately label all plots and include any requested code listings with your solutions. \emph{Only include those scripts and functions which are requested}. Show your work for problems that you do by hand. A clear and complete presentation of your solutions is required for full credit.

\begin{questions}

\question \emph{Jacobi Method}.
\begin{parts}
\part Write a function with the declaration \verb$function x = jacobi(A,x0,b,tol)$ which implements the Jacobi method for solving a linear system of equations. Its inputs are the matrix \verb$A$, an initial guess \verb$x0$, the right-hand-side \verb$b$, and an absolute error tolerance \verb$tol$. Its output \verb$x$ is an approximate solution of the linear system. Include a listing of your function.
\part Consider the matrices stored in \verb$jacobi1.mat$, \verb$jacobi2.mat$, and \verb$jacobi3.mat$. Are any of these matrices strictly diagonally dominant? With this information, what can you conclude about the convergence of the Jacobi method?
\part Use the \verb$eig$ command to determine the spectral radius of the iteration matrix in each case. With this information, what can you conclude about the convergence of the Jacobi method?
\part Use the Jacobi method to solve the linear system of equations in each case, with a right-hand-side so that the solution is $(1,1,\dots,1)$. Use the initial guess $(0,0,\dots,0)$ and the tolerance $10^{-6}$. For each of the three matrices, record the true error at each iteration (using the known solution). Plot the error as a function of iteration number on a single set of axes using the \verb$semilogy$ command. You should find that the matrices each require a different number of iterations. Explain your findings. What does this plot tell you about the convergence rate of the Jacobi Method?
\end{parts}

\question \emph{Gauss-Seidel Method}. This is another stationary iterative method with (perhaps) enhanced convergence properties. The iteration is given by
\[\vec{x}_{k+1} = D^{-1}\big(\vec{b}-U\vec{x}_k-L\vec{x}_{k+1}\big)\]
which means that you will need to solve a lower triangular system of equations for $\vec{x}_{k+1}$ at each step.

\begin{parts}
\part Write a function with the declaration \verb$function x = gauss_seidel(A,x0,b,tol)$ which implements the Gauss-Seidel method for solving a linear system of equations. Its inputs and outputs are the same as in the previous problem. Include a listing of your function. \emph{Important}: Do not find the inverse of the matrix $D+L$. Instead, use the \verb$\$ operator to solve using forward substitution. 
\part  Repeat part (d) of the previous problem using the Gauss-Seidel method. For each of the three matrices, record the true error at each iteration (using the known solution). Plot the error as a function of iteration number on a single set of axes using the \verb$semilogy$ command. You should find that the matrices each require a different number of iterations. Explain your findings. What does this plot tell you about the convergence rate of the Gauss-Seidel method?
\part Compare the convergence properties of the Gauss-Seidel method with the Jacobi method. What are some of the pros and cons of using one method versus the other?
\end{parts}

\end{questions}

\end{document}
