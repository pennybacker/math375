\documentclass[12pt]{exam}

\usepackage{amssymb,amsfonts,amsmath}
\usepackage[letterpaper,margin=1in]{geometry}
\usepackage{graphicx}

\newcommand{\class}{MATH 375}
\newcommand{\term}{Spring 2015}
\newcommand{\doctitle}{Homework 2}

\newcommand{\matlab}{{\sc Matlab}}

\parindent 0ex

\pagestyle{head}
\header{\bf \class}{\bf \doctitle\ - Page \thepage\ of \numpages}{\bf \term}
\headrule

\renewcommand{\arraystretch}{1.5}

\begin{document}

Remember to adequately label all plots and include any \matlab~scripts and functions with your solutions. A clear and complete presentation of your solutions is required for full credit.
\begin{questions}

\question Convert the following base-2 numbers to decimal: $1011.\overline{011}$, $11.101\overline{10}$.

\question Sauer, Section 0.3, Exercises 1(b), 1(c), 8.

\question Sauer, Section 0.4, Computer Exercise 3.

\question Write a \matlab~function to solve the quadratic equation $ax^2 + bx + c = 0$ using the classical quadratic formula
\[x_\pm = \frac{-b \pm \sqrt{b^2 - 4ac}}{2a} .\]

\begin{parts}
\part Test your function on the following cases:

\begin{subparts}
\subpart $a = 2$, $b = 3$, $c = 1$.
\subpart $a = 1$, $b = 3$, $c = 4$.
\end{subparts}
In each case and for choice of sign, verify the accuracy of the numerical solution by comparing it with the exact solution.
\part Now consider the case $a = 1$, $b = 3$, $c = 8^{-14}$. One solution is $x_- \approx -3$. Test your function on this case. It should perform poorly, so modify it as described in Example 0.6 to obtain accurate approximate roots. Finally, make a table which displays three sets of roots for this case: those calculated with your original function, those calculated with your modified function, and those calculated with \matlab's built-in \verb$roots$ function.
\end{parts}

\end{questions}

\end{document}
